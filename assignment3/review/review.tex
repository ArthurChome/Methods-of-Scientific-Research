\documentclass[14]{article}

% Used packages
\usepackage{apacite}
\usepackage[12pt]{moresize}
%\usepackage[numbers,sort&compress]{natbib}
\usepackage{blindtext}
\usepackage{enumitem}
\usepackage{xcolor}
\usepackage[explicit,noindentafter]{titlesec}
\usepackage[rightcaption]{sidecap}
\usepackage{caption}
\usepackage{float}
\usepackage{csquotes}
\usepackage{titling}
\usepackage{titlesec}
\usepackage{graphicx} %package to manage images
\graphicspath{{images/}}



\titlespacing\section{1pt}{12pt plus 0pt minus 0pt}{6pt plus 0pt minus 0pt}
\titlespacing\subsection{1pt}{12pt plus 4pt minus 2pt}{6pt plus 2pt minus 2pt}
\titlespacing\subsubsection{1pt}{12pt plus 4pt minus 2pt}{6pt plus 2pt minus 2pt}
\titlespacing\paragraph{1pt}{12pt plus 4pt minus 2pt}{6pt plus 2pt minus 2pt}

% New commands
\newcommand{\addsection}[3]{\addtocontents{toc}{\protect\contentsline{section}{\protect\numberline{#1}#2}{#3}}}
\newcommand{\addsubsection}[3]{\addtocontents{toc}{\protect\contentsline{subsection}{\protect\numberline{#1}#2}{#3}}}
\newcommand{\itab}[1]{\hspace{0em}\rlap{#1}}
\newcommand{\tab}[1]{\hspace{.5\textwidth}\rlap{#1}}
\newcommand{\Csh}{C{\lserif\#}}


\begin{document}
\author{\textbf{Faculty of Sciences and Bio-Engineering Sciences}\\[2\baselineskip]\newline\textbf{Arthur Chomé - 0529279}}

\date{ \LARGE Assignment 3: Review}
\title{\vspace{-6cm}}%\underline{Everything is connected}}

\maketitle

\section{Introduction}
Based on our student number (0529279), we had the option between paper 2, 4 and 9 to write our review about. We went for paper number 4\cite{dandekar2013biased} discussing the issue of opinion polarisation within society. 
%\newline 
This phenomenon is brough about by social factors such as homophily --interaction between like-minded individuals-- and biased assimilation whereupon individuals --when confronted with inconclusive evidence on a complex issue-- draw undue support to their own position on the matter. Noticeable is the tendency for individuals to directly accept information defending their own standings but to reject it when not enforcing their own personal beliefs. Furthermore, homophily on its own is not sufficient to cause polarization but it can be the case when coupled with biased assimilation.
\newline
After a brief introduction to the issue of polarisation and their contributing factors, it mentions DeGroot's mathematical model\cite{degroot1974reaching} used for the explanation of divergence in opinions. It criticizes the built model for not including biased assimilation all the while proposing an improvement on the model that includes this social factor. From the start, the paper informed us of the mathematical capabilities for consensus reaching and its applications such as measuring the degree of disagreement within a group. It's a surprise to read that even the most complex social problems can be mathematically expressed as models and explained using graph theory.

\section{Opinion}
The paper's subject is interesting --trying to mathematically explain the social problem of polarisation-- but it lacks the necessary touch to make it attractive for members of any two categories for a number of reasons. Its introduction might deter pure mathematical researchers of reading further but the content following upon it is too advanced for the average social scientist to follow.
\newline
If we were only to base ourselves on the introduction of the paper --its writing style and title-- we might convince ourselves that it was written for a target audience with no explicit mathematical background such as political scientists or psychologists. However, the paper shifts directly towards a mathematical focus introducing rather complex notions about graph theory and models. It also lists pure definitions to the reader while not providing any additional explanations or figures to clarify the context. It demands a certain level of expertise of the reader not previously made clear from the introduction. 
\newline
Another fault of the paper is its lack of a clear structure. Though it provides an introduction, the separation between related work, experiments and conclusion is not sufficient. When reading the paper, I had trouble distinguishing previous work from the paper's contribution. It also does not help that paragraphs and definitions in the paper have the same heading making for more confusion.
\newline
The model uses three recommender algorithms as to calculate the relative disagreement between members of a node graph: SALSA\cite{lempel2001salsa}, Personalized PageRank\cite{page1999pagerank}, and item-based collaborative filtering\cite{linden2003amazon} respectively. Though they are properly referred, they only provide mathematical formulas to explain the workings of each algorithm. 

\section{Improvements}
For the above critique on the paper, we have some suggestions for improvements to be made. This would in turn increase the relative attractiveness of the paper increasing its influence. The following suggestions are focusing on the writing the style, the paper's structure and the discussion of algorithms.

\subsection{Content}
The paper's introduction focuses too much on the terminology and concepts of polarization. It is misleading to be afterwards discussing the paper's extension of DeGroot's mathematical consensus model for the explanation of divergence in opinions. A suggestion is to shorten the paper's introduction on polarization and focus more on the mathematical model they developed for it. The title of the paper should be reworked, it does not stick with the actual content of the paper.

\subsection{Writing Style}
Though the paper has a good introduction on the topic and clearly states its contributions, everything after it is too formulaic and dry for people without a decent knowledge on graphy theory to understand. I suggest to remove the definitions in the paper in exchange for a more high-level, easy-to-understand explanation all the while referring to papers citing these definitions. The use of figures should not be excluded to 

\subsection{Algorithm Comparison}
The inclusion of pseudocode\footnote{\protect\url{https://en.wikipedia.org/wiki/Pseudocode}} for each algorithm should not be missed. It provides a high-level description of each of their operation.
\newline 
Although all three algorithms operate by ways of random walks on a node graph G, there is a performance comparison to be made. One of the biggest concerns in Computer Science is to perform work as efficient as possible without wasting resources. This is why a table of the best, worst and average case performance of each algorithm should be provided in Big-O notation\footnote{\protect\url{https://en.wikipedia.org/wiki/Big\_O\_notation}}

%From the perspective of a Computer Science student, the algorithms referenced in this paper are not clearly explained in terms of workings or application. Adding pseudocode or other code snippets could alleviate this instead of using the pure mathematical formulas. Another problem

%proposes a mathematical model based on DeGroot's model\cite{degroot1974reaching} that also accounts for biased assimilation. proposed a process where at each time step, individuals simultaneously update their opinion to the weighted average of their neighbors’ and their own opinionat the previous time step.


%the paper offers a mathematical model as to measure the extend to which opinions within a group can diverge. It is represented as a graph of nodes representing persons and their opinions on a subject (e.g. religion, freedom of speech,...) with the graph's edges the relationships between them. For each relationship, there is an added weight representing the degree to which two nodes influence each other's opinion. As to measure the degree in which the opinions differ, the paper formulated a Network Disagreement Index.
%\newline 
%DeGroot\cite{degroot1974reaching} proposed a process where at each time step, individuals simultaneously update their opinion to the weighted average of their neighbors’ and their own opinion at the previous time step.

%The degree to which opinions diverge in this matter is made clear 

%It offers a formation process for biased opinion formation process.
%Polarisation has increased in American society brought about by recent technology trends such as the internet, talk shows and cable news. Reason for this is that individuals are being confronted with a high rate of diverging opinions enabling them to select information conforming to their own standings leading to more extreme beliefs.

%Contributing factors to this are homophily within groups and biased assimilation. Homophily refers to the  

%Polarization in this sense of the word refers to an increased divergence of opinions within individuals of a group. opinion formation process is polarizing if it results in increased divergence of opinions.



%\subsection{Graph Theory}

%\paragraph{Strengths}

%\paragraph{Weaknesses}

%\section{Improvements}

%\section{Conclusion}

\bibliographystyle{apacite}
\bibliography{bibliography}	
\end{document}
